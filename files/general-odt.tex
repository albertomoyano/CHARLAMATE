\documentclass[12pt,a4,oneside,onecolumn,openany,final]{book}
\usepackage{polyglossia}
\setmainlanguage{spanish}
\usepackage[autostyle=true]{csquotes}

% diseño de listas
\usepackage{enumitem}
\setlist{nosep,topsep=-\parskip}

% diseño ISO690
\usepackage[style=iso-authoryear,
backend=biber,
sortcites=true,
indexing=cite,
texencoding=utf8,
bibencoding=utf8]{biblatex}

\defcounter{biburlnumpenalty}{3000}
\defcounter{biburlucpenalty}{6000}
\defcounter{biburllcpenalty}{9000}

% Generar los índices, desactivamos las opciones para títulos
\renewbibmacro*{bibindex}{%
  \ifbibindex
    {\indexnames{author}%
     \indexnames{editor}%
     \indexnames{translator}%
     \indexnames{commentator}}
    {}}

\renewbibmacro*{citeindex}{%
  \ifciteindex
    {\indexnames{author}%
     \indexnames{editor}%
     \indexnames{translator}%
     \indexnames{commentator}}
    {}}

% figuras y epígrafe
\usepackage{graphicx}
\usepackage{svg}
\usepackage[labelfont=bf,font=small,labelsep=period,format=plain]{caption}

% Cita con cambio DE márgenes
\renewenvironment{quote}
{\normalsize\list{}{\leftmargin=14pt \rightmargin=0pt}%
\item\relax}
{\endlist}

% Nuevo tipo de entorno flotante para fotografías
\usepackage{newfloat}
\DeclareFloatingEnvironment[
fileext=lop,
listname={Índice de imágenes},
name=Imagen,
placement=ht,
%within=section,% activate it if you want
%chapterlistsgaps=on,% meaningful only if chapters exist
]{imagen}

% configuración de los valores decimales de los cuadros
\usepackage{siunitx}
\sisetup{output-decimal-marker={.},
group-separator={\,},
group-minimum-digits=3,
table-text-alignment=center,
detect-all}

\usepackage{xcolor}

%% se deshabilitan momentaneamente hasta encontrar el error de inicial
% generamos los índices
\usepackage[xindy]{imakeidx}
\makeindex
\makeindex[name=names,title={Índice de autoras y autores}]
\makeindex[name=concepto,title={Índice de conceptos}]
\makeindex[name=onomastico,title={Índice onomástico}]

% generamos el índice para biblatex
\DeclareIndexNameFormat{default}{%
\usebibmacro{index:name}{\index[names]}
 	{\namepartfamily}
 	{\namepartgiven}
 	{\namepartprefix}
 	{\namepartsuffix}}
\renewbibmacro*{citeindex}{%
 	\ifciteindex
 	{\indexnames{labelname}}
{}}

% generamos los glosarios
\usepackage[acronym,sanitizesort,toc=true,nonumberlist]{glossaries}
\preto\chapter{\glsresetall}
\makenoidxglossaries
\renewcommand{\glsnamefont}[1]{\sf\textbf{\textup{#1}}}

% Diseño y estilo a glosario y acrónimo
\newglossarystyle{crossreflist}%
{% base it on list (adapt as required)
	\renewcommand*{\glossentry}[2]{%
		\item[\glsentryitem{##1}%
		\glstarget{##1}{\glossentryname{##1}}]
		\glossentrydesc{##1}\glspostdescription\space ##2%
		% check if the user1 key has been supplied:
		\ifglshasfield{useri}{##1}%
		{% do cross-reference
			\newline
			\glsletentryfield{\crossrefs}{##1}{useri}%
			\glsseeformat[\emph{Véase:}]{\crossrefs}{}%
		}%
		{}%
	}%
}
\setglossarystyle{crossreflist}

% estos paquetes deben ir al final
\usepackage{url}

% Diseño de la raya del medio
\makeatletter
\def\thinskip{\hskip 0.16667em\relax}
\def\endash{--}
\def\emdash{\endash-}
\def\d@sh#1#2{\unskip#1\thinskip#2\thinskip\ignorespaces}
\def\dash{\d@sh\nobreak\endash}
\def\Dash{\d@sh\nobreak\emdash}
\def\ldash{\d@sh\empty{\hbox{\endash}\nobreak}}
\def\rdash{\d@sh\nobreak\endash}
\def\Ldash{\d@sh\empty{\hbox{\emdash}\nobreak}}
\def\Rdash{\d@sh\nobreak\emdash}
\def\hyph{-\penalty\z@\hskip\z@skip}
\def\slash{/\penalty\z@\hskip\z@skip}
\makeatother

\usepackage{ifthen}

% fin del preambulo
