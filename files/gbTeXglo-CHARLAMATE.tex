
\newglossaryentry{@glo194-clai}{
type = \acronymtype,
name         = {CLAI},
description  = {Consejo Latinoamericano de Iglesias},
first        = {Consejo Latinoamericano de Iglesias (CLAI)},
text         = {CLAI},
}
\newglossaryentry{@glo195-coas}{
type = \acronymtype,
name         = {COAS},
description  = {Corriente de Organización y Acción Sindical},
first        = {Corriente de Organización y Acción Sindical (COAS)},
text         = {COAS},
}
\newglossaryentry{@glo200-latex}{
name         = {LaTeX},
description  = {es un sistema de composición de tipografía de alta calidad; incluye características diseñadas para la producción de documentación técnica y científica. LaTeX es la norma de facto para la comunicación y publicación de documentos científicos. Está disponible como software libre},
text         = {LaTeX},
}
\newglossaryentry{@glo201-ceheal}{
type = \acronymtype,
name         = {CEHEAL},
description  = {Centro de Estudios de Historia Económica Latinoamericana y Argentina},
first        = {Centro de Estudios de Historia Económica Latinoamericana y Argentina (CEHEAL)},
text         = {CEHEAL},
}
\newglossaryentry{@glo202-joseingenieros}{
name         = {José Ingenieros},
description  = {(nacido como Giuseppe Ingegnieri,​ Palermo, 24 de abril de 1877 - Buenos Aires, 31 de octubre de 1925) fue un médico, psiquiatra, psicólogo, criminólogo, farmacéutico, sociólogo, filósofo, masón, teósofo, ​escritor y docente ítaloargentino. Su libro \emph{Evolución de las ideas argentinas} marcó rumbos en el entendimiento del desarrollo histórico de Argentina como nación. Se destacó por su influencia entre los estudiantes que protagonizaron la Reforma Universitaria de 1918},
text         = {José Ingenieros},
}